%----------------------------------------------------------------------------------------
%	ABSTRACT PAGE
%----------------------------------------------------------------------------------------

\begin{abstract}
\addchaptertocentry{\abstractname} % Add the abstract to the table of contents
Bezpilotní letadla (UAV) nacházejí široké uplatnění v~oblastech jako je průzkum, monitorování prostředí nebo doprava. Klíčovým problémem pro jejich autonomní využití je nedostatečná přesnost při přistávání, což může být zlepšeno pomocí přídavných lokalizačních zařízení, která jsou však často nákladná a složitá na nastavení. Tato diplomová práce se zaměřuje na automatizaci přistávání UAV se svislým přistáním pomocí kamery a fiduciárního markeru v~simulačním prostředí za působení větru a stínění plošiny.

Jsou prozkoumány různé metody přistávání a použitelnost několika simulátorů pro simulaci přistávání čtyřrotorového letadla a vnějších vlivů, navržen simulační systém, s~jehož využitím jsou experimentálně zjištěny vlastnosti 4 implementovaných metod vzhledem k~vnějším vlivům. Pokročilé metody zvýšily robustnost a snížily počet pokusů potřebných k~přistání v~nepříznivých podmínkách. Stínění plošiny má vliv na maximální výšku, ze které lze úspěšně a spolehlivě detekovat marker a navržený simulační systém umožňuje efektivní testování různých metod přistávání za různých vnějších podmínek a zachytávání jejich výsledků, což podporuje rychlou iteraci při vývoji nových algoritmů.
\vspace{20pt}
\begin{center}\textbf{\ttitleeng}\end{center}
Unmanned Aerial Vehicles (UAVs) have diverse applications such as exploration, environmental monitoring, and transportation, yet landing automation remains a challenge due to imprecise pose estimation. Precision can be improved with additional localization devices, however these devices are often expensive and complex to set up. This thesis focuses on automating the vertical landing of UAVs using a camera and a fiducial marker in a simulated environment under the influence of wind and platform shading.

Various landing methods and the usability of several simulators for simulating the landing of a quadrotor aircraft and external influences are researched. A~simulation system is designed and utilized to experimentally determine the properties of 4 implemented methods concerning external influences. Advanced methods increased robustness and reduced the number of attempts needed for landing in adverse conditions. Platform shading affects the maximum height from which the marker can be successfully and reliably detected. The proposed simulation system enables effective testing of different landing methods under various external conditions and capturing their results, supporting rapid iteration in the development of new algorithms.
\end{abstract}