        %%%%%%%%%%%%%%%%%%%%%%%%%%%%%%%%%%%%%%%%%
% Masters/Doctoral Thesis 
% LaTeX Template
% Version 2.3 (25/3/16)
%
% This template has been downloaded from:
% http://www.LaTeXTemplates.com
%
% Version 2.x major modifications by:
% Vel (vel@latextemplates.com)
%
% This template is based on a template by:
% Steve Gunn (http://users.ecs.soton.ac.uk/srg/softwaretools/document/templates/)
% Sunil Patel (http://www.sunilpatel.co.uk/thesis-template/)
%
% Template license:
% CC BY-NC-SA 3.0 (http://creativecommons.org/licenses/by-nc-sa/3.0/)
%
%%%%%%%%%%%%%%%%%%%%%%%%%%%%%%%%%%%%%%%%%

%----------------------------------------------------------------------------------------
%	PACKAGES AND OTHER DOCUMENT CONFIGURATIONS
%----------------------------------------------------------------------------------------

\documentclass[
11pt, 						% The default document font size, options: 10pt, 11pt, 12pt
oneside, 					% Two side (alternating margins) for binding by default, uncomment to switch to one side
chapterinoneline,			% Have the chapter title next to the number in one single line
czech, 					    % ngerman for German
singlespacing, 				% Single line spacing, alternatives: onehalfspacing or doublespacing
% draft, 					% Uncomment to enable draft mode (no pictures, no links, overfull hboxes indicated)
nolistspacing, 				% Uncomment this to set spacing in lists to single
%liststotoc, 				% Uncomment to add the list of figures/tables/etc to the table of contents
%toctotoc, 					% Uncomment to add the main table of contents to the table of contents
parskip, 					% Uncomment to add space between paragraphs
%nohyperref, 				% Uncomment to not load the hyperref package
headsepline, 				% Uncomment to get a line under the header
hidelinks,
]{style/thesis_style} 		% The class file specifying the document structure

\usepackage[utf8]{inputenc} % Required for inputting international characters
\usepackage[T1]{fontenc} 	% Output font encoding for international characters

\usepackage{lmodern} 		% Font family

% Use the bibtex backend with the authoryear citation style (which resembles APA) (citestyle=alphabetic)
\usepackage[backend=bibtex,style=numeric,citestyle=authoryear-comp,natbib=true,sorting=ynt,block=ragged]{biblatex} 
\addbibresource{bib/thesis_bibliography.bib} 				% The filename of the bibliography

% Required to generate language-dependent quotes in the bibliography
\usepackage[autostyle=true]{csquotes}

%----------------------------------------------------------------------------------------
%	MARGIN SETTINGS
%----------------------------------------------------------------------------------------

\geometry{
	paper=a4paper, 			% Change to letterpaper for US letter
	inner=1.1cm, 			% Inner margin
	outer=3cm,				% Outer margin
	bindingoffset=2cm, 		% Binding offset
	top=1.5cm, 				% Top margin
	bottom=1.5cm, 			% Bottom margin
	%showframe,				% show how the type block is set on the page
}

% --------------------------------------
% Kitt added style and packages
% --------------------------------------
\usepackage[table,xcdraw]{xcolor}
\definecolor{lightgray}{gray}{0.95}
\definecolor{lightblue}{rgb}{0.6,0.75,0.85}
\definecolor{lightyellow}{rgb}{0.95,1.0,0.8}

\definecolor{guired}{RGB}{237,28,36}
\definecolor{guiorange}{RGB}{255,127,39}
\definecolor{guigreen}{RGB}{34,177,76}
\definecolor{guiblue}{RGB}{0,162,232}

\usepackage{adjustbox}

\usepackage{courier}

\usepackage{etoolbox}

\usepackage{slashbox}

\usepackage{tikz}
\usepackage{pgfplots}
\pgfplotsset{compat=1.8}

\usepackage{listofitems} % for \readlist to create arrays
\usetikzlibrary{positioning, fit, arrows.meta, shapes, babel, shapes.geometric} % for arrow size
\usepackage[outline]{contour} % glow around text
\contourlength{1.4pt}

\tikzset{>=latex} % for LaTeX arrow head
\usepackage{xcolor}
\colorlet{myred}{red!80!black}
\colorlet{myblue}{blue!80!black}
\colorlet{mygreen}{green!60!black}
\colorlet{myorange}{orange!70!red!60!black}
\colorlet{mydarkred}{red!30!black}
\colorlet{mydarkblue}{blue!40!black}
\colorlet{mydarkgreen}{green!30!black}
\tikzstyle{node}=[thick,circle,draw=myblue,minimum size=22,inner sep=0.5,outer sep=0.6]
\tikzstyle{node in}=[node,green!20!black,draw=mygreen!30!black,fill=mygreen!25]
\tikzstyle{node hidden}=[node,blue!20!black,draw=myblue!30!black,fill=myblue!20]
\tikzstyle{node convol}=[node,orange!20!black,draw=myorange!30!black,fill=myorange!20]
\tikzstyle{node out}=[node,red!20!black,draw=myred!30!black,fill=myred!20]
\tikzstyle{connect}=[thick,mydarkblue] %,line cap=round
\tikzstyle{connect arrow}=[-{Latex[length=4,width=3.5]},thick,mydarkblue,shorten <=0.5,shorten >=1]
\tikzset{ % node styles, numbered for easy mapping with \nstyle
  node 1/.style={node in},
  node 2/.style={node hidden},
  node 3/.style={node out},
}
\def\nstyle{int(\lay<\Nnodlen?min(2,\lay):3)} % map layer number onto 1, 2, or 3
\usepackage{tikzit}
\input{img/style.tikzstyles}

\newcommand{\empt}[2]{$#1^{\langle #2 \rangle}$}


\newcommand{\code}[1]{\texttt{#1}}
\newcommand{\file}[1]{\code{#1}}

\usepackage{listings}
\lstset{frame=single, basicstyle=\footnotesize\ttfamily, backgroundcolor=\color{lightgray}}
\renewcommand{\lstlistingname}{Část kódu}
\renewcommand{\lstlistlistingname}{Seznam částí kódu}

\usepackage{amsmath}
\usepackage{amsfonts}
\usepackage{bm}
\usepackage{hyperref}
\usepackage[]{cleveref}
\providecommand\crefpairconjunction{ a\nobreakspace}
\providecommand\crefrangeconjunction{ až }
\providecommand\creflastconjunction{\crefpairconjunction}
\providecommand\crefpairgroupconjunction{\crefpairconjunction}
\crefname{listing}{část kódu}{části kódu}
\crefname{figure}{obrázek}{obrázky}
\crefname{table}{tabulka}{tabulky}
\crefname{equation}{rovnice}{rovnice}
\crefname{chapter}{kapitola}{kapitoly}
\crefname{section}{sekce}{sekce}
\crefname{enumerate}{bod}{body}
\renewcommand{\crefpairconjunction}{ a\nobreakspace}
\renewcommand{\crefrangeconjunction}{ až }
\renewcommand{\creflastconjunction}{\crefpairconjunction}
\renewcommand{\crefpairgroupconjunction}{\crefpairconjunction}

\newcommand{\refskl}[2]{\hyperref[#1]{#2 \labelcref{#1}}}

\usepackage{float}

\usepackage{epstopdf}

\usepackage{commath}

\usepackage{multirow}

\usepackage{url}

\newcommand{\pluseq}{\mathrel{+}=}

\usepackage{caption}
\usepackage{subcaption}

\usepackage{tocloft}
\addtolength{\cftchapnumwidth}{5pt}

\makeatletter
\def\l@figure{\@dottedtocline{1}{1.5em}{3em}}
\makeatother

\makeatletter
\renewcommand\paragraph
  {%
    \@startsection{paragraph}{4}{\z@}{3.25ex \@plus 1ex \@minus .2ex}{-1ex}
      {\normalfont\normalsize\bfseries}
  }
\makeatother

\usepackage{pdfpages}

\usepackage[most]{tcolorbox}

\usepackage{enumitem}

\usepackage{forest}

\usepackage[useregional,showseconds=false]{datetime2}

\usepackage{longtable}
\usepackage{booktabs}
\usepackage{bigstrut}
\renewcommand{\arraystretch}{1.2} % more space between the rows

\usepackage{array} % for defining a new column type
\usepackage{varwidth} %for the varwidth minipage environment
\newcolumntype{M}{>{\begin{varwidth}{4cm}}l<{\end{varwidth}}}

\def\xstrut{\rule[-2ex]{0pt}{5ex}}

\usepackage[automake,acronym]{glossaries}

\renewcommand*{\acronymname}{Seznam zkratek}

\makeglossaries

% \newacronym{dps}{DPS}{Deska plošných spojů}
% \newacronym{mcu}{MCU}{Microcontroller Unit (jednočipový počítač)}
% \newacronym{url}{URL}{Uniform Resource Locator}
\newacronym{gui}{GUI}{Graphical User Interface - grafické uživatelské rozhraní}
% % \newacronym{chmu}{ČHMÚ}{Český hydrometeorologický ústav}
% \newacronym{ffnn}{FFNN}{Feedforward Neural Network (dopředná neuronová síť)}
\newacronym{lstm}{LSTM}{Long Short-Term Memory - druh rekurentní neuronové sítě}
% \newacronym{mse}{MSE}{Mean Squared Error (střední kvadratická odchylka)}
\newacronym{api}{API}{Application Programming Interface - rozhraní pro programování aplikací}
% \newacronym{nn}{NS}{Neuronová síť}
% \newacronym{relu}{ReLU}{Aktivační funkce Rectified Linear Unit}
% \newacronym{iot}{IoT}{Internet of Things (internet věcí)}
% \newacronym{af}{AF}{Aktivační funkce}

\newacronym{uav}{UAV}{Unmanned Aerial Vehicle - bezpilotní letadlo}
\newacronym{mse}{MSE}{Mean Squared Error - střední kvadratická chyba}
\newacronym{mae}{MAE}{Mean Absolute Error - střední absolutní chyba}
\newacronym{gps}{GPS}{Global Positioning System - globální polohový systém}
\newacronym{imu}{IMU}{Inertial Measurement Unit - inerciální měřicí jednotka}
\newacronym{udp}{UDP}{User Datagram Protocol}
\newacronym{hil}{HIL}{Hardware in the loop}
\newacronym{sih}{SIH}{Simulation in Hardware - simulace na hardwaru}
\newacronym{sw}{SW}{Software}
\newacronym{esc}{ESC}{Electronic Speed Controller - elektronický kontrolér rychlosti}
\newacronym{rtk}{RTK}{Real-time Kinematic}

% \includeonly{tex/chap/chap_hardware.tex,tex/chap/chap_data_collection.tex,tex/chap/chap_features,tex/chap/chap_regressors.tex,tex/chap/chap_gui.tex,tex/chap/chap_eval.tex,tex/chap/chap_discussion.tex,tex/chap/chap_conclusion.tex,tex/chap/chap_introduction.tex}
% \includeonly{tex/chap/chap_introduction.tex,tex/chap/chap_features.tex}
\newcommand{\draftfig}{false}

\setcounter{tocdepth}{2}

\preto\tabular{\shorthandoff{-}}

\usepackage{afterpage}

\DeclareRobustCommand{\kolecko}[1]{%
	\begin{tikzpicture}[baseline=-1mm]
		\node [draw,circle,node font=\tiny] {#1};
	\end{tikzpicture}%
}

\usepackage{keystroke}