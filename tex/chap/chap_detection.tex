\chapter{Detekce fiduciárních markerů} \label{chap:detection}
  Během detekce fiduciárních značek se využívá zejména algoritmů počítačového vidění za účelem zvýraznění a extrakce informace související s tagem a naopak potlačení pozadí a také případně nalezení a lokalizace klíčových bodů, které mohou sloužit pro odhad polohy značky v prostoru v případě známého rozměru tagu a kalibrované kamery. Jako detekce se označuje proces nalezení značky v obrazu, algoritmus, který toto provádí se pak nazývá detektor.

  Mezi algoritmy používané při detekci fiduciárních markerů patří prahování, vyhledávání vzorů (template matching), hranové detektory (\cite{apriltag2}) nebo Houghova transformace (\cite{Shabalina2019}). Dále je popsáno, jak se některé z těchto metod uplatňují v detektoru použitém v praktické části (Apriltag).

  \paragraph{Apriltag} Detektor apriltagů se skládá z pěti dílčích kroků: 1. adaptivního prahování, 2. segmentace souvislých hranic, 3. aproximace čtyřúhelníků, 4. rychlého dekódování a 5. volitelného zpřesnění hran. Během prahování se volí práh vždy lokálně v závislosti na okolí prahovaného bodu jako aritmetický průměr minimálního a maximálního jasu. Pro snížení výpočetní náročnosti je v tomto kroku vstupní obraz navíc decimován tak, že se extrémy hledají vždy v buňkách po 4x4 pixelech, ve verzi 2 (\cite{apriltag2}) se autoři chtějí vyhnout nespojitostem na hranicích buněk a tak používají extrémy z osmiokolí (po buňkách). Ve verzi 3 (\cite{apriltag3}) potom zjistili, že je vhodnější decimaci oddělit od prahování a používají bodové převzorkování, které lépe zachová hrany v obrazu, ale může vést na aliasing. Zachování hran je vhodné, protože se v dalším koroku detekují. Málo kontrastní body, které vzejdou z prahování se v dalším zpracování neuvažují, čímž jsou další kroky zrychleny. %TODO: přepracovat na subparagraph, doplnit obrázek algoritmu

  Hranice se segmentují tak, že se nejprve najdou hrany v prahovaném obrazu, čili takové pixely, které sousedí s pixelem opačné barvy, které se následně slučují do hranic pomocí algoritmu union-find. Problém, kdy jsou dva velmi blízké tagy odděleny pouze tenkou linií bílých pixelů, který by při sloučení hranic znemožňoval detekovat oba tagy, je vyřešen tak, že bílé pixely mohou být součástí 2 hranic. Každé takto segmentované části hranice je přiřazeno nějaké číslo unikátní v rámci obrázku (tzv. obarvení). (\cite{apriltag2}) Zrychlení ve verzi 3 je docíleno tím, že se zbytečně nesjednocují již sjednocené části (ušetří se volání sjednocení), navíc se včasně zamítají příliš malé oblasti, které by nemohly vést na dekódovatelný tag. (\cite{apriltag3})

  Pro každé sjednocení z předchozího kroku je potřeba najít vhodný čtyřúhelník, tzn. rozdělit množinu hraničních bodů do 4 skupin, které odpovídají jednotlivým stranám čtyřúhelníku. Hledání optimálního řešení je příliš výpočetně náročné, proto se postupuje tak, že se naleznou kandidáti na rohy a poté se projdou všechny jejich kombinace. Rohy se hledají tak, že se body nejprve seřadí podle úhlu průvodiče vedeného z centroidu množiny a jednotlivými body, poté se postupně aproximují body v posuvném okně přímkou a hledají se maxima střední kvadratické chyby (\acrshort{mse}), odpovídající body jsou prohlášeny za rohy. Pro každou podmnožinu 4 rohů se zbylé body rozdělí na strany a každá z nich se aproximuje přímkou. Vybere se taková kombinace rohů, pro kterou je \acrshort{mse} nejnižší. (\cite{apriltag2}) Ve verzi 3 se řazení bodů provádí podle kvadrantu, ve kterém se bod nachází a sklonu průvodiče. Není tak nutné počítat explicitně úhel. (\cite{apriltag3})

  Nalezené čtyřúhelníky se vyrovnají (pomocí homografie vypočtené ze souřadnic polohy jejich rohů) a hledá se nejbližší kód z dané rodiny značek pro každou ze 4 možných rotací. Omezí-li se počet odlišných bitů na 2, je možné předpočítat všechny tagy maximálně 2 bity vzdálené od validního tagu a zaznamenat je do hashovací tabulky. Při detekci se pak z tabulky jen vybere příslušná hodnota, nebo se detekce zamítne. (\cite{apriltag2}) Verze 3 zavádí bilineární interpolaci buněk tagu a zvyšuje osttrost za účelem zlepšení detekce malých značek. (\cite{apriltag3})