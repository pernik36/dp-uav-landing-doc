\chapter{Experimenty a vyhodnocení} \label{chap:eval}
Poznámky k provedení: U všech experimentů několik tříd podmínek (bezvětří, slabý vítr, slabý vítr s poryvy, silný vítr, silný vítr s poryvy; stín 0\% 30\%, 50\%, 70\%, 100\%) a porovnat algoritmy. Chtěl bych se domluvit na prioritě pokusů a některé nechat až na konec, kdyby zbyl čas.
\section{Přesnost přistání}
  Sledovat chybu v jednotlivých proměnných po dosednutí.
\section{Přesnost přistávání}
  Sledovat chybu sledování trajektorie v průběhu přistávání (porovnat MSE?)
\section{Vliv stínu na podíl nedetekovaných markerů}
  Počítat podíl snímků, ve kterých nebyl detekován tag po zahájení přistávání.
\section{Úspěšnost přistávání}
  Porovnat (střední) počet pokusů nutných k úspěšnému dosednutí.
\section{Ověření nezávislosti výsledků na směru větru}
  Aby se ušetřil počet pokusů, ověřil bych jen za silného větru, zda se nějak liší průběh přistávání.
\section{Ověření nezávislosti výsledků na vzájemné počáteční poloze dronu a plošiny}
\section{Odhad kovariančních matic pro Kálmánův filtr}
  Nevím, jestli vůbec podrobněji rozepisovat zde, když už je to popsáno v \refskl{sec:kalmanoffboardpid}{podkapitole}.
\section{Střední doba výpočtu v jednom kroku algoritmu}
  Tento experiment mě nenapadl, ale pro celkové porovnání algoritmů by byl přínosný.