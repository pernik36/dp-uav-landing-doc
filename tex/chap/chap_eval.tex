\chapter{Experimenty a vyhodnocení} \label{chap:eval}
  V navrženém systému byly implementovány a odsimulovány 4 vybrané metody přistávání (\cref{chap:algs}), na nichž byly provedeny experimenty popsané v této kapitole, které sloužily primárně ke zjištění rozdílů mezi implementovanými algoritmy podle různých hledisek. Během několikrát opakované simulace byly sledovány určité ukazatele, které byly zaznamenávány a statisticky zpracovány, konkrétní provedení je vždy uvedeno v příslušné podkapitole daného experimentu. Pro provedení experimentů \cref*{sec:presnostPristani,sec:presnostPristavani,sec:uspesnost} bylo zvoleno 5 různých tříd větrných podmínek (bezvětří, slabý vítr, slabý vítr s poryvy, čerstvý vítr, čerstvý vítr s poryvy). Nastavení simulačních parametrů v jednotlivých třídách shrnuje \cref{tab:tridyVetru}.
  \begin{table}
    \centering
    \begin{tabular}{@{}lllll@{}}
      \hline
      třída & $v$ [m/s] & $\sigma_v$ [m/s] & $\varsigma$ [$^{\circ}$] & $\sigma_\varsigma$ [$^{\circ}$] \bigstrut\\
      \hline
      bezvětří & 0     & 0     & 0     & 0 \bigstrut[t]\\
      slabý vítr & 3     & 1     & 105   & 10 \\
      slabý vítr s poryvy & 3     & 3     & 105   & 20 \\
      čerstvý vítr & 8     & 2     & 105   & 10 \\
      čerstvý vítr s poryvy & 8     & 4     & 105   & 20 \bigstrut[b]\\
      \hline
      \end{tabular}%
    \caption[Třídy větrných podmínek]{Třídy větrných podmínek a jejich parametry použité při simulaci v některých experimentech.}
    \label{tab:tridyVetru}
  \end{table}
  \section{Přesnost přistání} \label{sec:presnostPristani}
    Podstatou tohoto experimentu bylo zjistit, jak přesně \acrshort{uav} dosedne na plošinu, použijí-li se jednotlivé algoritmy. Simulace byla provedena 50x pro každou z 20 dvojic a zaznamenávala se skutečná poloha letounu v simulačním prostředí v okamžiku jeho dosednutí. Během vyhodnocení se porovnávala tato zaznamenaná poloha s polohou plošiny dle definice dané mise. Z rozdílů v souřadnicích $x, y$ a rotaci $\psi$ (kolem osy $z$) byly pro každý z případů vypočteny výběrová střední hodnota a výběrový rozptyl. Shrnutí takto zjištěných výsledků obsahuje \cref{tab:presnostPristani} a v grafické podobě zobrazující vizualizaci střední hodnoty, kovarianční elipsy a každého z přistání přímo na obrázku plošiny \cref{fig:presnostPristani}.
  \section{Přesnost přistávání} \label{sec:presnostPristavani}
    Další experiment zjišťoval přesnost celého přistávacího manévru, kdy během každé ze simulací postupně počítal střední absolutní chybu (\acrshort{mae}) v souřadnicích $x, y$ a rotaci $\psi$ v každém kroku algoritmu vůči jím požadované trajektorii (ve 2 případech přímka kolmo procházející středem plošiny, ve 2 případech skloněná přímka pod stejným úhlem jako v daném větru přistávající dron nasměrovaná proti němu). Stejně jako v předchozím případě byl pokus proveden 50x pro každý prvek kartézského součinu metody přistávání $\times$ třídy větru. \Cref{tab:presnostPristavani} shrnuje výběrové střední hodnoty a výběrové rozptyly \acrshort{mae} na konci přistávání pro dané podmínky simulace.
  \section{Úspěšnost přistávání} \label{sec:uspesnost}
    Během silných poryvů se stává, že se letoun v blízkosti plošiny vychýlí natolik, že se marker na plošně dostane mimo zorné pole kamery. V takový okamžik ztrácí algoritmus možnost podle obrazu odhadovat vzájemnou polohu \acrshort{uav} a značky, proto čeká 50 snímků s požadavkem na fixní polohu na případné odregulování poruchy způsobené poryvem, po kterém by mohl být marker opět viditelný, a mezitím stoupá rychlostí 0{,}3~m/s. Neobjeví-li se během této doby, považuje se pokus o dosednutí za neúspěšný a je opakován tak, že dron podle \acrshort{gps} přelétává na přibližnou polohu plošiny do výšky 10~m a sleduje, jestli se objeví na snímcích z kamery marker. Po úspěšné detekci se opět pokouší přistát. Při tomto experimentu se zaznamenával počet pokusů o dosednutí a vypočetla se jeho střední hodnota přes 50 provedených pokusů u všech dvojic metod přistávání a tříd větru. Výsledky jsou v \refskl{tab:uspesnost}{tabulce}.
  \section{Střední doba výpočtu v jednom kroku algoritmu} \label{sec:stredniDobaVypoctu}
    Pro posouzení výpočetní náročnosti jednotlivých metod byla v průběhu přistávání zaznamenávána také průměrná doba potřebná k výpočtu jednoho kroku algoritmu od obdržení snímku z kamery po předání řídicího požadavku letové řídicí jednotce. Krok algoritmu zahrnuje detekci fiduciárních markerů, odhad jejich polohy, přepočet polohy do vodorovné souřadné soustavy dronu, volitelně filtraci Kálmánovým filtrem, přepočet na odchylku od vybrané trajektorie a výpočet požadavku na rychlosti pomocí PID regulátorů. Střední doba výpočtu jednoho kroku algoritmu je pro jednotlivé metody uvedena v \refskl{tab:doba}{tabulce}.
  \section{Vliv stínu na podíl nedetekovaných markerů}
    Počítat podíl snímků, ve kterých nebyl detekován tag po zahájení přistávání.
  % \section{Ověření nezávislosti výsledků na směru větru}
  %   Aby se ušetřil počet pokusů, ověřil bych jen za silného větru, zda se nějak liší průběh přistávání.
  % \section{Ověření nezávislosti výsledků na vzájemné počáteční poloze dronu a plošiny}
  % \section{Odhad kovariančních matic pro Kálmánův filtr}
  %   Nevím, jestli vůbec podrobněji rozepisovat zde, když už je to popsáno v \refskl{sec:kalmanoffboardpid}{podkapitole}.