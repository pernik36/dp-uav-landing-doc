\chapter{Diskuze} \label{chap:discussion}
% Discussion helps to make your paper memorable - consolidate (summarize) your data to make it easier to remember. Briefly summarize your Methods (what you did) and Results (what you got and what you learnt).

% Skeleton:
% \begin{itemize}
%     \item Recap your Methods
%     \item Recap your Results
%     \item List other researchers’ reports using similar methods
%     \item Compare you results to theirs
%     \item \begin{itemize}
%         \item List similarities and differences and/or
%         \item Make a prediction and/or
%         \item Describe a relevant empirical rule
%     \end{itemize}
% \end{itemize}

% Poznámky:
% U \refskl{sec:presnostPristavani}{experimentu} by bylo dobré porovnat nejen metody, ale i ladění jejich PID.
% V \refskl{sec:mise}{podkapitole} je řečno, že se nevyužívá nastavení strany plošiny. Proč?

Jedním z~cílů této práce bylo navrhnout systém pro simulaci přistání \acrshort{uav} na plošině, která byla navržena pro navádění letadla pomocí vidění tak, že ji pokrýval fiduciární marker. Tento způsob navádění byl zvolen kvůli jednoduchosti, kdy nevyžaduje žádný speciální hardware, výhodou je i možnost použití v~uzavřeném prostoru. Z několika možností fiduciárních markerů byl zvolen AprilTag, který svým rekurzivním uspořádáním 2~tagů umožnil větší rozsah vzdáleností, ve kterých mohl být detekován.

Ústřední součástí navrženého systému je simulátor. Před jeho výběrem byla provedena rešerše, která se zabývala podporovanými funkcemi simulátorů, aby mohly být naplněny cíle práce, a letovými řídicími softwary, aby byl zjednodušen případný přenos na reálné letadlo, z~níž jako kandidáti vzešly Gazebo a AirSim. První volba zvítězila kvůli jejímu tradičnímu použití v~podobných úlohách a rozsáhlé komunitě, ale při vývoji systému se ukázalo, že nemá zcela vyspělou dokumentaci, a oživení některých součástí vyžadovalo mnoho neúspěšných pokusů. Spolupráce s~dalšími programovými prostředky je také do jisté míry omezená, v~čemž se při zpětném pohledu zdá být AirSim lepší. Kromě toho má také realističtější renderování, ale nevýhodou je, že jeho vývoj byl přerušen. Na poli letových řídicích \acrshort{sw} připadaly v~úvahu PX4 a ArduPilot, které byly oba podporovány oběma kandidátními simulátory, a byla zvolena PX4 jako profesionálnější možnost, jejíž autoři se přímo podílejí na vývoji používaných komunikační protokolů a byla tak očekávána jejich dobrá podpora.

Byly simulovány 2 vnější vlivy, konkrétně vliv větru v~5 různých třídách rychlosti a proměnlivosti rychlosti i směru a vliv zastínění plošiny. Teplota vzduchu nebyla simulována i přes to, že může mít výrazný vliv na výkon akumulátoru dronu, protože Gazebo tento vliv neumí promítnout a vlastní implementace modelu V~navrženém systému by bylo možné po úpravě simulovat i další vlivy, například kouř nebo mlhu v~prostředí. V~praktické části byly provedeny experimenty, které zjišťovaly, jak tyto vnější vlivy působí na navržené přistávací algoritmy.

Systém pro simulaci, který byl v~této práci představen a implementován, umožňuje kromě simulace přistávání řešit i další obecnější úlohy, protože je pro algoritmy definované jednotné rozhraní a je možné dodat jiný vlastní algoritmus. S~uživatelem systém komunikuje pomocí grafického rozhraní, umožňujícího připravovat mise s~různými parametry, spouštět je samostatně, nebo dávkově spolu s~dalšími, přičemž se jejich výsledky uchovají ve strukturovaných textových souborech. Průběh mise včetně pohledu z~kamery a stavu probíhajícího algoritmu zahrnujícího například aktuální chybu vůči požadované trajektorii je možné sledovat na zvláštní obrazovce v~tomto rozhraní.

Práce se dále zabývala metodami přistávání uvedenými v~související literatuře včetně přistávání na pohyblivých plošinách nebo plošinách s~dynamickými značkami. Pro praktickou část byly navrženy 4 příbuzné metody, které byly podrobeny experimentům za různých podmínek a bylo zaznamenáno jakou mají přesnost přistání, přesnost sledování zvolené trajektorie, úspěšnost, jak dlouho přistání trvá, jaká je jejich výpočetní náročnost a jaký vliv na detekci značek má stín, který překrývá část plošiny.

Z~hlediska přesnosti přistávání mezi implementovanými metodami nebyly velké rozdíly a kvůli vlastnostem simulátoru, ve kterém se uplatňovalo příliš malé tření, byl dron i po přistání snášen dále od středu plošiny. Větší vliv na přesnost přistávání by mohlo mít ladění PID regulátoru s~využitím dynamického modelu letadla a případně i modelu neurčitosti, které kvůli rozsahu nebylo provedeno. Mohlo by se jednat o~důležité rozšíření pro budoucí práci.

Přestože vylepšení základního algoritmu v~podobě skloněné přistávací trajektorie a Kálmánova filtru samostatně vylepšují úspěšnost přistávání i jeho rychlost a v~některých případech vylepšují přesnost přistávání, jejich kombinace v~jedné metodě nepřináší další zlepšení a zejména v~přesnosti je často i horší. Může to být způsobeno tím, že kombinace do smyčky programu přidává další zátěž, čímž se do systému zavede významné dopravní zpoždění a požadavky pro dron kvůli němu opozdí natolik, že to má vliv na dynamiku řízení letadla, protože poruchy nejsou dostatečně rychle odregulovány.

Výhodou pokročilejších metod (zejména těch s~přistáváním po skloněné přímce) bylo to, že jejich řídicí strategie byly schopné i v~čerstvém větru s~poryvy přistát průměrně na \landingRetriesMeanLAPILvitrIV. pokus bez Kálmánova filtru, resp. \landingRetriesMeanKAPILvitrIV. pokus s~Kálmánovým filtrem, což je méně než polovina pokusů, jež průměrně potřebovala metoda, která používala pouze proporcionální regulátor.

%TODO: Zastínění.
