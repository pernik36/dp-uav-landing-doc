\chapter{Návrh systému pro simulaci přistávání \acrshort{uav} na plošině}
  Jedním z cílů této práce bylo navrhnout systém, pomocí kterého by bylo možné simulovat přistání bezpilotního letounu na plošině s využitím některé metody z \refskl{chap:algs}{kapitoly}. Jeho návrh je obsahem této kapitoly. Byla zvolena částečně modulární architektura, díky čemuž je možné testovat různé metody přistávání nebo i jiné úlohy. Systém využívá vlastní komponenty i komponenty od jiných autorů, které jsou dále popsány v \refskl{sec:components}{podkapitole} včetně důvodů, proč byly zvoleny. \refskl{sec:inputs,sec:outputs}{Podkapitoly} popisují, jaké vstupy systém očekává a jakým způsobem jsou využity v komponentách, co je jeho výstupem a odkud daná informace pochází. Dále (\refskl{sec:structure}{podkapitola}) je na systém nahlíženo zevnitř a jsou charakterizovány vazby mezi komponenty a jejich informační obsah.
  
  \section{Komponenty systému} \label{sec:components}
    \subsection{Letový řídicí software}
      Letový řídicí software běží při jeho použití v realitě přímo na počítači, který je umístěný na palubě dronu a k němuž jsou obvykle připojeny všechny senzory a aktuátory, jimiž je letoun vybaven. Využití přímo tohoto softwaru při simulaci zmenšuje rozdíly mezi simulačním prostředím a realitou, což umožňuje snadnější přenos systému do skutečnosti při jeho případném reálném nasazení.
      
      Senzory a aktuátory jsou při simulaci pouze napodobené simulátorem a software běží buď stejně jako simulátor v použitém osobním počítači, nebo, je-li to podporováno, může být spuštěn na skutečném palubním počítači \acrshort{uav}, se zavedenými umělými vstupy ze simulace (tzv. \acrfull{sih}). Software zpracovává data ze senzorů a vstupy od uživatele, generuje akční zásahy, které předává aktuátorům a může poskytovat telemetrické služby nebo služby vysokoúrovňového dálkového ovládání např. prostřednictvím protokolu MAVLink.
      
      Právě takové služby se využívají v navrhovaném systému a z toho důvodu byl z výběru uvedeného v \refskl{chap:sims}{kapitole} vyřazen řídicí software BetaFlight, který je možné řídit jen modelářským rádiovým ovladašem. Ze zbývajících dvou variant byl v návrhu dále uvažován kontrolér PX4, který je zaměřený více na profesionální a akademické využití a také nabízí širší možnosti podporovaných simulátorů než ArduPilot.
    \subsection{Simulátor}
      Simulátor je v navrhovaném systému počítačový program, který napodobuje realitu, letovému řídicímu softwaru poskytuje umělá měření ze senzorů v simulovaném prostředí, přijímá od něj výstupy pro aktuátory a modeluje jejich vliv na simulované vozidlo. Dále simuluje dynamické chování celého letadla i ostatních předmětů umístěných napodobeném světě a, podporuje-li to, i jejich vzájemné interakce. Také zprostředkovává vizuální pohled do této umělé reality prostřednictvím simulované kamery.

      Z možností podporovaných kontorlérem PX4 bylo vybráno Gazebo, které alespoň částečně podporuje všechny aspekty, které byly při návrhu považovány za důležité, je rozšiřitelné, má bohatou komunitu a je pro podobné úlohy běžně používáno. Ostatním simulátorům některé vlastnosti chyběly, měly jiné zaměření, skončila jim podpora nebo nebyly příliš rozšířené. I přes to by dalším dobrým kandidátem byl AirSim, který má bohatší \acrshort{api} než Gazebo s dobrou dokumentací, takže by absence velké komunity nemusela působit problémy. Má také realističtější renderování.
    \subsection{\acrshort{gui}}
      Grafické uživatelské rozhraní (\acrshort{gui}) je vlastní komponentou navrhovaného systému a vytváří jednotné prostředí pro ovládání a nahlížení stavu ostatních komponent uživatelem. Tím tvoří rozhraní mezi uživatelem a systémem. Pro jeho konstrukci byl zvolen framework Qt s bindingy do jazyka Python PyQt (\cite{pyqt}), který kromě grafického rozhraní propojuje i některé komponenty v systému. O funkcích \acrshort{gui} podrobně pojednává \cref{chap:gui}. 
    \subsection{Hodiny ze simulátoru}
      Některé kroky přistávacích metod je nutné časově opozdit. Simulační čas se ale může lišit od reálného. Z toho důvodu je jednou z komponent systému také přijímač hodinového signálu ze simulátoru, který ostatním komponentám může poskytovat časovač. Ten se aktivuje po uplynutí stanovené doby v simulačním čase a může danou komponentu asynchronně informovat o této události. Čas ze simulátoru Gazebo se přenáší pomocí zpráv prostřednictvím knihovny gz-transport a Python bindingů gz-python \cite{gz-python}. 
    \subsection{Kamera ze simulátoru}
      Pro odhad polohy plošiny vzhledem k letounu se používá obraz prostředí s fiduciárním markerem, jejž je nutné získat ze simulátoru. To se provádí stejným způsobem jako u hodin, tedy pomocí gz-python, jen s využitím zprávy jiného typu. Kamera je v simulátoru Gazebo reprezentována pluginem, který obrazová data zprostředkovává jako zprávy, a její definice je součástí modelu letounu. Před použitím je nutné kameru kalibrovat, což bylo provedeno zachycením několika desítek obrázků čtvercové plochy rozdělené na černé a bílé čtverce o známých rozměrech uspořádaných do šachovnice z různých úhlů. Později se ukázalo, že informace o kalibraci je možné získat přímo z pluginu tím, že se definuje téma zpráv, do kterého se vysílá tato i další metadata.

      Modul navrhovaného systému, který přijímá obrazová data ze simulátoru zároveň provádí i jejich zpracování pomocí detektoru fiduciárních markerů, který je také komponentou tohoto systému a je popsán dále. Metodám přistávání tento modul zprostředkovává detektorem vypočtenou relativní vzdálenost v jednotlivých osách, rotaci ve svislé ose a obraz s označenými detekovanými značkami.
    \subsection{Detektor fiduciárních markerů}
      V obrazu ze simulátoru se detekují fiduciární značky a odhaduje se podle nich vzájemná poloha plošiny a \acrshort{uav}, jehož součástí je kamera. Tyto dílčí úlohy zajišťuje právě detektor fiduciárních markerů (\cref{chap:detection}). Pro použití v navrhovaném systému byly vybrány značky AprilTag 3, protože umožňují konstrukci rekurzivních značek a tím i detekci ve větším rozsahu vzdáleností. K jejich rozpoznávání se využívá knihovna AprilTag 3 (\cite{apriltag3}) s Python bindingy dt-apriltag (\cite{dt-apriltags}).
    \subsection{Metoda přistávání}
      Metodou přistávání se rozumí ta komponenta navrhovaného systému, která kombinuje informace z různých jiných součástí (např. senzory z řídicího \acrshort{sw}, polohová data z kamery atp.) za účelem ovládání dronu dle postupu popsaného v \refskl{chap:algs}{kapitole}. Jedná se tedy o implementaci metody přistávání. \Acrshort{uav} je ovládáno pomocí protokolu MAVLink, konkrétně prostřednictvím jeho implemetace v mavsdk pro Python (\cite{mavsdk}).
    \subsection{Mise a jejich seznamy}
      Uživatel může pomocí \acrshort{gui} nebo v YAML souboru definovat vybrané podmínky prostředí a další vstupy systému (\refskl{sec:inputs}{podkapitola}) pro jeden průběh zvoleného přistávacího algoritmu. Takové definici se říká mise a je možné jich určit více. Všechny definice se ukládají v souboru, takže mezi běhy systému mohou být uchovány. Mise je navíc možné zahrnovat do jejich seznamů, jež se nazývají experimenty, a spouštět je postupně automaticky s vybraným počtem opakování pro každou položku zvlášť. Uchování definovaných experimentů je řešeno stejným způsobem jako u misí, tedy uložením do souboru. Každá mise má po svém ukončení určitý výsledek, který se v případě spuštění mise jako součásti experimentu uchovává ve výstupním souboru společně s výsledky již proběhlých misí.
    \subsection{Model světa}
      Umělé prostředí je definováno modelem světa ve formátu SDF (\cite{sdf}), který je využíván při simulaci. Obsahuje plochu s leteckým snímkem Borského parku v Plzni, definici světelných podmínek, vlastnosti větru a model přistávací plošiny a její stínění. Pro každou misi se podle vstupů dodaných uživatelem generuje odpovídající soubor, který je pak předán simulátoru.

  \section{Vstupy systému} \label{sec:inputs}
    \subsection{Počáteční poloha \acrshort{uav} a plošiny}
      Do simulovaného světa se před začátkem simulace umisťuje model letounu a plošiny, jejichž poloha v rovině země vzhledem k počátku simulovaného světa (jeho geografické souřadnice jsou dány v konfiguračním souboru a odpovídají středu leteckého snímku použitého jako podkladu) je dána dvěma souřadnicemi v metrech (v osách $x$ a $y$) a úhlem rotace ve stupních kolem svislé osy $z$. Těchto 6 vstupů (3 pro \acrshort{uav} a 3 pro plošinu) se zadává při definici mise v \acrshort{gui} a jsou také součástí uložené mise v souboru (\cref{sec:saved}). Kvůli omezení simulátoru a způsobu jeho spouštění neexistuje přímočarý způsob, jak nastavit rotaci modelu dronu před spuštěním simulace, proto se rotoce nastavuje až za letu stejným způsobem, jakým se zadávají příkazy ke změně polohy prostřednictvím řídicího \acrshort{sw} \acrshort{uav}.
    \subsection{Nastavení zastínění plošiny}
      Jedním ze simulovaných vnějších vlivů je zastínění plošiny, které je dáno 2 vstupy - podílem zastíněné a nezastíněné délky úsečky vedené kolmo na stranu plošiny, která míří ke zdroji světla, jejím středem a sklonem stínu ve stupních. Oba parametry jsou součástí definice mise (soubor nebo \acrshort{gui}). Stínění je napodobováno umístěním tenkého kvádru o rozměrech 30~x~5~m (šířka~x~výška) vysoko nad povrch na takové místo, aby vznikl požadovaný stín. Gazebo nesimuluje rozptyl světla postupně procházejícího atmosférou, takže ostrost stínu je závislá na přednastaveném konstantním rozptylu světla, ale ne na vzdálenosti vrhajícího objektu od vrženého stínu.
    \subsection{Nastavení větru}
      Vítr je simulován pomocí pluginu WindEffects, jehož parametry jsou nastaveny na základě 4 vstupních hodnot zadaných v \acrshort{gui}, nebo definici mise v souboru. Nastavuje se vodorovná rychlost a její směrodatná odchylka v metrech za sekundu a úhel směru větru vzhledem k ose $x$ a jeho směrodatná odchylka ve stupních.
    \subsection{Výběr přistávací metody}
      Metody přistávání jsou předdefinované včetně jejich parametrů (např. zisky složek PID regulátoru) a jsou identifikovány názvem. Při definici mise v \acrshort{gui} si uživatel vybere jednu z metod z rozbalovací nabídky, v souboru s definicemi misí je pak uložen její název. Při spuštění mise se pak volají metody této vybrané metody.
    \subsection{Uložoné mise a jejich seznamy} \label{sec:saved}
      Po spuštění systému se dříve definované a uložené mise a jejich seznamy (tzv. experimenty) načtou ze souborů \texttt{mise.yaml}, který obsahuje seznam definic misí a \texttt{experimenty.yaml}, který obsahuje seznam definic experimentů. Definice experimentu je seznam názvů misí a počtu jejich opakování v rámci experimentu.
    \subsection{Konfigurační soubor}
      Konfigurační soubor obsahuje mnoho dalších vstupů, které jsou potřebné pro správný běh systému. Jedná se mimo jiné o nastavení některých cest k souborům a složkám (např. složka pro uchovávání definic umělých světů), výchozí kalibrace kamery, která se může použít dokud není doručena kalibrace přímo ze simulátoru, model \acrshort{uav}, který se má použí pro simulaci, kovarianční matice a matice modelu letounu pro Kálmánův filtr, směr dopadajícího slunečního záření nebo geografické souřadnice počátku kartézského souřadného systému umělého světa.
  \section{Výstupy systému} \label{sec:outputs}
    \subsection{3D vizualizace simulovaného prostředí}
      Simulátor Gazebo má volitelně grafický režim, jehož součástí je 3D vizualizace simulace, ve které je možné se volně pohybovat a sledovat, co se ve světě odehrává.
    \subsection{Pohled kamery s vyznačenými detekovanými markery}
      Ve vlastním \acrshort{gui} se v záložce \uv{Živě} zobrazuje pohled z kamery, který je využívaný řídicím algoritmem přistávání. V obrazu jsou červeným čtyřúhelníkem vyznačeny všechny AprilTagy, které detektor zachytil, a jejich souřadnice a rotace v souřadné soustavě kamery.
    \subsection{Stav mise a experimentu}
      V záložce \uv{Živě} uživatelského rozhraní se také vyskytuje textový souhrn stavu probíhajícího experimetu a mise. Zobrazuje se aktuální opakování dané mise a její pořadí v probíhajícím experimetu. Přistávací metody jsou členěny do několika stavů (\cref{fig:algObecny}), z nichž má každý název, který se na téže záložce také zobrazuje. % TODO: Kromě toho se během přistávání zobrazují také grafy skutečné a zdánlivé chyby polohy \acrshort{uav} vzhledem k požadované trajektorii.
    \subsection{Výsledek mise a experimentu}
      Po ukončení mise se její výsledek, který zahrnuje různé sledované metriky, %TODO: dodat po implemetaci
      vypisuje do konzole a v případě, že mise byla spuštěna jako součást experimentu se navíc výsledek přidá do souboru s výsledky všech doposud proběhlých misí experimentu. Tento soubor je pak možné po ukončení analyzovat a vyvodit souhrnné výsledky.
    \subsection{Uložené mise a jejich seznamy}
      Při definici mise nebo experimentu v \acrshort{gui} se na disk uloží soubor, který zahrnuje všechny definice misí, respektive experimentů, aby mohly být při dalším spuštění systému načteny.
  \section{Struktura systému} \label{sec:structure}
    \begin{figure}
      \ctikzfig{img/sysStruktura}
      \caption[Struktura navrhovaného systému]{Struktura navrhovaného systému pro simulaci přistávání \acrshort{uav}, jeho komponenty a jejich vzájemné informační vazby, vstupy a výstupy}
      \label{fig:offboardpidangle}
    \end{figure}